%%%%%%%%%%%%%%%%%%%%%%%%%%%%%%%%%%%%%%%%%
% Machine Learning @ PUC-Rio 2017.2
%%%%%%%%%%%%%%%%%%%%%%%%%%%%%%%%%%%%%%%%%
% Ian Albuquerque Raymundo da Silva - 1310451
% Clara de Mattos Szwarcman - 1310351
%%%%%%%%%%%%%%%%%%%%%%%%%%%%%%%%%%%%%%%%%

%%%%%%%%%%%%%%%%%%%%%%%%%%%%%%%%%%%%%%%%%
% Journal Article
% LaTeX Template
% Version 1.4 (15/5/16)
%
% This template has been downloaded from:
% http://www.LaTeXTemplates.com
%
% Original author:
% Frits Wenneker (http://www.howtotex.com) with extensive modifications by
% Vel (vel@LaTeXTemplates.com)
%
% License:
% CC BY-NC-SA 3.0 (http://creativecommons.org/licenses/by-nc-sa/3.0/)
%
%%%%%%%%%%%%%%%%%%%%%%%%%%%%%%%%%%%%%%%%%

%----------------------------------------------------------------------------------------
%	PACKAGES AND OTHER DOCUMENT CONFIGURATIONS
%----------------------------------------------------------------------------------------

\documentclass[twoside,twocolumn]{article}

\usepackage{blindtext} % Package to generate dummy text throughout this template 
\usepackage[utf8]{inputenc}
\usepackage{graphicx}

\usepackage[sc]{mathpazo} % Use the Palatino font
\usepackage[T1]{fontenc} % Use 8-bit encoding that has 256 glyphs
\linespread{1.05} % Line spacing - Palatino needs more space between lines
\usepackage{microtype} % Slightly tweak font spacing for aesthetics

\usepackage[english]{babel} % Language hyphenation and typographical rules

\usepackage[hmarginratio=1:1,top=32mm,columnsep=20pt]{geometry} % Document margins
\usepackage[hang, small,labelfont=bf,up,textfont=it,up]{caption} % Custom captions under/above floats in tables or figures
\usepackage{booktabs} % Horizontal rules in tables

\usepackage{lettrine} % The lettrine is the first enlarged letter at the beginning of the text

\usepackage{enumitem} % Customized lists
\setlist[itemize]{noitemsep} % Make itemize lists more compact

\usepackage{abstract} % Allows abstract customization
\renewcommand{\abstractnamefont}{\normalfont\bfseries} % Set the "Abstract" text to bold
\renewcommand{\abstracttextfont}{\normalfont\small\itshape} % Set the abstract itself to small italic text

\usepackage{titlesec} % Allows customization of titles
\renewcommand\thesection{\Roman{section}} % Roman numerals for the sections
\renewcommand\thesubsection{\roman{subsection}} % roman numerals for subsections
\titleformat{\section}[block]{\large\scshape\centering}{\thesection.}{1em}{} % Change the look of the section titles
\titleformat{\subsection}[block]{\large}{\thesubsection.}{1em}{} % Change the look of the section titles

\usepackage{fancyhdr} % Headers and footers
\pagestyle{fancy} % All pages have headers and footers
\fancyhead{} % Blank out the default header
\fancyfoot{} % Blank out the default footer
\fancyhead[C]{Supervised Learning with Fashion MNIST $\bullet$ December 2017 $\bullet$ Machine Learning @ PUC-Rio} % Custom header text
\fancyfoot[RO,LE]{\thepage} % Custom footer text

\usepackage{titling} % Customizing the title section

\usepackage{hyperref} % For hyperlinks in the PDF

%----------------------------------------------------------------------------------------
%	TITLE SECTION
%----------------------------------------------------------------------------------------

\setlength{\droptitle}{-4\baselineskip} % Move the title up

\pretitle{\begin{center}\Huge\bfseries} % Article title formatting
\posttitle{\end{center}} % Article title closing formatting
\title{Supervised Learning with Fashion MNIST} % Article title
\author{%
\textsc{Clara de Mattos Szwarcman}\thanks{Clara's student ID: 1310351} \\[1ex] % Your name
\normalsize Pontifícia Universidade Católica do Rio de Janeiro \\ % Your institution
\normalsize \href{mailto:clara_szw@hotmail.com}{clara\_szw@hotmail.com} % Your email address
\and % Uncomment if 2 authors are required, duplicate these 4 lines if more
\textsc{Ian Albuquerque Raymundo da Silva}\thanks{Ian's student ID: 1310451} \\[1ex] % Second author's name
\normalsize Pontifícia Universidade Católica do Rio de Janeiro \\ % Second author's institution
\normalsize \href{mailto:ian.albuquerque.silva@gmail.com}{ian.albuquerque.silva@gmail.com} % Second author's email address
}
\date{\today} % Leave empty to omit a date
\renewcommand{\maketitlehookd}{%
\begin{abstract}
\noindent \blindtext % Dummy abstract text - replace \blindtext with your abstract text
\end{abstract}
}

%----------------------------------------------------------------------------------------

\begin{document}

% Print the title
\maketitle

%----------------------------------------------------------------------------------------
%	ARTICLE CONTENTS
%----------------------------------------------------------------------------------------

\section{Introduction}

\lettrine[nindent=0em,lines=3]{T} he MNIST dataset is considered the "hello world" of Machine Learning.
Current results achieve over 99.7\% accuracy using different techniques
\cite{LWan:2013dg} \cite{DCirean:2012dg} \cite{ISato:2015dg} \cite{JRChang:2015dg} \cite{CYLee:2015dg}.
Also, many good techniques do not work well on this dataset because of its simplicity \cite{CFran:2017dg}.
With that in mind, the Fashion MNIST dataset \cite{FashionMNIST} was created. It has same size and format
of the MNIST dataset (70.000 grayscale images of size 28x28 labeled into 10 different classes) meaning that
it should be as simple to use, but its images consist of fashion items which are supposedly harder to
classify.

The goal of this work is to explore different techniques of supervised learning for classifying the images
in the Fashion MNIST dataset. We want to measure the accuracy of our tests and compare them with different
results on the same dataset, while also trying to obtain some intuition behind the data.

%----------------------------------------------------------------------------------------

\section{Preparation}

The Fashion MNIST is already subdivided into two sets. A set of 60.000 images (the training set)
and a set of 10.000 images (the test set). The idea is to learn with the training set and then
evaluate the performance of our trained model with the test set.

We will be using the standard accuracy as a metric for our tests:
\begin{equation}
\label{eq:acc}
accuracy = \frac{\#(correct\_classifications)}{\#(test\_set\_size)}
\end{equation}

The dataset is already provided as a list of (28*28)+1 sized vectors corresponding
to each of the 28*28 pixels of the image plus 1 entry for the classification.
Since each pixel value was an integer from 0 to 255, we have opted to divide
each value by 255 to achieve features that goes from 0 to 1.

%----------------------------------------------------------------------------------------

\section{Known Results}

The Fashion MNIST repository contains some benchmarks using the python library
scikit-learn \cite{scikitlearn} \cite{scikitlearnbenchmark}. They are not meant
to be efficient or good, but they work as a baseline for our project.
The best result (corresponding to the best selection of parameters tested)
of each algorithm are displayed in table \ref{table:scikit-benchmark}.

\begin{table}
\centering
\begin{tabular}{llr}
\toprule
Algorithm Name & Best Accuracy (\%) \\
\midrule
SVC & $89.7$ \\
Gradient Boosting & $88.8$ \\
Random Forest & $87.9$ \\
MLP & $87.7$ \\
K-Neighbors & $86.0$ \\
Logistic Regression & $84.0$ \\
SGD & $82.9$ \\
Decision Tree & $80.1$ \\
\bottomrule
\end{tabular}
\caption{Fashion MNIST Scikit-Learn Benchmark \cite{scikitlearnbenchmark} }
\label{table:scikit-benchmark}
\end{table}

The current state of art for the Fashion MNIST dataset is a recent one, with 96.3\% accuracy using Wide
Residual Networks using random erasing data augmentation \cite{randomerasingdataaugmentationpaper}.
We will consider this as the ceiling accuracy for our project.

%------------------------------------------------

\section{Methods}

Maecenas sed ultricies felis. Sed imperdiet dictum arcu a egestas. 
\begin{itemize}
\item Donec dolor arcu, rutrum id molestie in, viverra sed diam
\item Curabitur feugiat
\item turpis sed auctor facilisis
\item arcu eros accumsan lorem, at posuere mi diam sit amet tortor
\item Fusce fermentum, mi sit amet euismod rutrum
\item sem lorem molestie diam, iaculis aliquet sapien tortor non nisi
\item Pellentesque bibendum pretium aliquet
\end{itemize}
\blindtext % Dummy text

Text requiring further explanation\footnote{Example footnote}.

%------------------------------------------------

\section{Approaches}

\subsection{Convolutional Neural Networks}

     The Convolutional Neural Networks (CNNs) are the current state-of-the-art for the Fashion-MNIST dataset with 96.3\% accuracy \cite{randomerasingdataaugmentationpaper}.For all tests with CNNs the values of the pixels were divided by 255, so that the values are always between 0 and 1. The first CNN that we used was based on the 1994s LeNet5 \cite{yannLeCun:1998}, one of the first CNNs, that has a 99.05\% accuracy for the MNIST dataset \cite{yannLeCun:mnist}. We used the same layers and kernel sizes but with more features on each convolutional layer. \ref{fig:lenet5} shows the architecture used for the first CNN test.

    \begin{figure}[h]
    \begin{center}
        \includegraphics[scale=0.8]{lenet5Arch.jpg}
        \caption{LeNet5 based Architecture\label{fig:lenet5}}
    \end{center}
    \end{figure}
    
    
    We trained the network for 200 epochs and got an accuracy of 89.02\%, then we added rotations from -5 to 5 degrees and got 89.81\%.
    
    In orde

\section{Results}

\begin{table}
\caption{Example table}
\centering
\begin{tabular}{llr}
\toprule
\multicolumn{2}{c}{Name} \\
\cmidrule(r){1-2}
First name & Last Name & Grade \\
\midrule
John & Doe & $7.5$ \\
Richard & Miles & $2$ \\
\bottomrule
\end{tabular}
\end{table}

\blindtext % Dummy text

\begin{equation}
\label{eq:emc}
e = mc^2
\end{equation}

\blindtext % Dummy text

%------------------------------------------------

\section{Discussion}

\subsection{Subsection One}

\blindtext % Dummy text

\subsection{Subsection Two}

\blindtext % Dummy text

%----------------------------------------------------------------------------------------
%	REFERENCE LIST
%----------------------------------------------------------------------------------------

\bibliography{references}
\bibliographystyle{plain}

%----------------------------------------------------------------------------------------

\end{document}
